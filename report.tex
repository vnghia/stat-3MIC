% Options for packages loaded elsewhere
\PassOptionsToPackage{unicode}{hyperref}
\PassOptionsToPackage{hyphens}{url}
%
\documentclass[
  11pt,
  xcolor = usenames,dvipsnames]{article}
\usepackage{amsmath,amssymb}
\usepackage{lmodern}
\usepackage{iftex}
\ifPDFTeX
  \usepackage[T1]{fontenc}
  \usepackage[utf8]{inputenc}
  \usepackage{textcomp} % provide euro and other symbols
\else % if luatex or xetex
  \usepackage{unicode-math}
  \defaultfontfeatures{Scale=MatchLowercase}
  \defaultfontfeatures[\rmfamily]{Ligatures=TeX,Scale=1}
\fi
% Use upquote if available, for straight quotes in verbatim environments
\IfFileExists{upquote.sty}{\usepackage{upquote}}{}
\IfFileExists{microtype.sty}{% use microtype if available
  \usepackage[]{microtype}
  \UseMicrotypeSet[protrusion]{basicmath} % disable protrusion for tt fonts
}{}
\makeatletter
\@ifundefined{KOMAClassName}{% if non-KOMA class
  \IfFileExists{parskip.sty}{%
    \usepackage{parskip}
  }{% else
    \setlength{\parindent}{0pt}
    \setlength{\parskip}{6pt plus 2pt minus 1pt}}
}{% if KOMA class
  \KOMAoptions{parskip=half}}
\makeatother
\usepackage{fancyvrb}
\usepackage{xcolor}
\IfFileExists{xurl.sty}{\usepackage{xurl}}{} % add URL line breaks if available
\IfFileExists{bookmark.sty}{\usepackage{bookmark}}{\usepackage{hyperref}}
\hypersetup{
  hidelinks,
  pdfcreator={LaTeX via pandoc}}
\urlstyle{same} % disable monospaced font for URLs
\VerbatimFootnotes % allow verbatim text in footnotes
\usepackage[margin=1in]{geometry}
\usepackage{listings}
\newcommand{\passthrough}[1]{#1}
\lstset{defaultdialect=[5.3]Lua}
\lstset{defaultdialect=[x86masm]Assembler}
\usepackage{longtable,booktabs,array}
\usepackage{calc} % for calculating minipage widths
% Correct order of tables after \paragraph or \subparagraph
\usepackage{etoolbox}
\makeatletter
\patchcmd\longtable{\par}{\if@noskipsec\mbox{}\fi\par}{}{}
\makeatother
% Allow footnotes in longtable head/foot
\IfFileExists{footnotehyper.sty}{\usepackage{footnotehyper}}{\usepackage{footnote}}
\makesavenoteenv{longtable}
\usepackage{graphicx}
\makeatletter
\def\maxwidth{\ifdim\Gin@nat@width>\linewidth\linewidth\else\Gin@nat@width\fi}
\def\maxheight{\ifdim\Gin@nat@height>\textheight\textheight\else\Gin@nat@height\fi}
\makeatother
% Scale images if necessary, so that they will not overflow the page
% margins by default, and it is still possible to overwrite the defaults
% using explicit options in \includegraphics[width, height, ...]{}
\setkeys{Gin}{width=\maxwidth,height=\maxheight,keepaspectratio}
% Set default figure placement to htbp
\makeatletter
\def\fps@figure{htbp}
\makeatother
\setlength{\emergencystretch}{3em} % prevent overfull lines
\providecommand{\tightlist}{%
  \setlength{\itemsep}{0pt}\setlength{\parskip}{0pt}}
\setcounter{secnumdepth}{5}
\usepackage{setspace}
\usepackage{float}
\usepackage{fontspec}
\usepackage{subfig}
\setmonofont{JetBrains Mono}[Contextuals=Alternate]
\floatplacement{figure}{H}
\lstset{ 
  language=R,                     % the language of the code
  basicstyle=\small\ttfamily, % the size of the fonts that are used for the code
  stepnumber=1,                   % the step between two line-numbers. If it is 1, each line
                                  % will be numbered
  numbersep=5pt,                  % how far the line-numbers are from the code
  backgroundcolor=\color{cyan!5},  % choose the background color. You must add \usepackage{color}
  showspaces=false,               % show spaces adding particular underscores
  showstringspaces=false,         % underline spaces within strings
  showtabs=false,                 % show tabs within strings adding particular underscores
  frame=single,                   % adds a frame around the code
  rulecolor=\color{black},        % if not set, the frame-color may be changed on line-breaks within not-black text (e.g. commens (green here))
  tabsize=2,                      % sets default tabsize to 2 spaces
  captionpos=b,                   % sets the caption-position to bottom
  breaklines=true,                % sets automatic line breaking
  breakatwhitespace=false,        % sets if automatic breaks should only happen at whitespace
  keywordstyle=\color{RoyalBlue},      % keyword style
  commentstyle=\color{Green},   % comment style
  stringstyle=\color{Orange},      % string literal style
}
\makeatletter
\renewcommand\paragraph{\@startsection{paragraph}{4}{\z@}%
        {-2.5ex\@plus -1ex \@minus -.25ex}%
        {1.25ex \@plus .25ex}%
        {\normalfont\normalsize\bfseries}}
\makeatother
\setcounter{secnumdepth}{4}
\hypersetup{
    colorlinks = true,
}
\ifLuaTeX
  \usepackage{selnolig}  % disable illegal ligatures
\fi

\author{}
\date{\vspace{-2.5em}}

\begin{document}

\onehalfspacing

\pagenumbering{gobble}

%\begin{titlepage}
\vspace*{\fill}
\begin{center}
\LARGE{\textbf{Analyse descriptive du jeu de données Spotify}}\\
\Large{\textbf{Projet en Statistique descriptive}}\\
\vspace*{1\baselineskip}
\Large{\textbf{Membres}}\\
LOULIDI Younes\\
PHAM Tuan Kiet\\
VO Van Nghia\\
\vfill % equivalent to \vspace{\fill}
\vspace*{\fill}
\Large{\textbf{Date}}\\
17 Mars, 2021
\end{center}
% \end{titlepage}

\newpage

\newpage
\renewcommand{\contentsname}{Table des matières}
\pagenumbering{roman}
\tableofcontents
\addcontentsline{toc}{section}{\contentsname}

\renewcommand{\listfigurename}{Table des figures}
\listoffigures

\newpage
\pagenumbering{arabic}
\doublespacing


\hypertarget{statistiques-descriptives-unidimensionnelle-et-bidimensionnelle}{%
\section{Statistiques descriptives unidimensionnelle et bidimensionnelle}\label{statistiques-descriptives-unidimensionnelle-et-bidimensionnelle}}

\hypertarget{la-nature-des-jeux-de-donnuxe9es}{%
\subsection{La nature des jeux de données}\label{la-nature-des-jeux-de-donnuxe9es}}

\hypertarget{des-jeux-de-donnuxe9es}{%
\subsubsection{Des jeux de données}\label{des-jeux-de-donnuxe9es}}

Ces jeux de données se composent de 10000 chansons extraites de la base de données Spotify.
Chaque ligne contient 11 variables statistiques comme suit:

\begin{itemize}
\tightlist
\item
  \passthrough{\lstinline!year!}: année de sortie du morceau,
\item
  \passthrough{\lstinline!acousticness!}: métrique relative interne de l'acoustique morceau,
\item
  \passthrough{\lstinline!duration!}: durée du morceau en millisecondes (ms),
\item
  \passthrough{\lstinline!energy!}: métrique relative interne de l'intensité, des rythmes du morceau,
\item
  \passthrough{\lstinline!explicit!}: vaut 1 si le morceau contient des vulgarités, et 0 sinon,
\item
  \passthrough{\lstinline!key!}: tonalité en début de morceau,
\item
  \passthrough{\lstinline!liveness!}: proportion du morceau où l'on entend un public,
\item
  \passthrough{\lstinline!loudness!}: mesure relative du volume du morceau (en décibels, dB)
\item
  \passthrough{\lstinline!mode!}: mode du morceau (0 si la tonalité est mineure, et 1 si la tonalité est majeure),
\item
  \passthrough{\lstinline!tempo!}: le tempo du morceau, en battement par minute (bpm),
\item
  \passthrough{\lstinline!pop.class!}: la popularité du morceau.
\end{itemize}

\hypertarget{des-variables-statistiques}{%
\subsubsection{Des variables statistiques}\label{des-variables-statistiques}}

Ici, nous précisons la nature de chaque variable et son format dans R.

\begin{longtable}[]{@{}llr@{}}
\toprule
Nom de variable statistique & Type de variable & Format dans R \\
\midrule
\endhead
\passthrough{\lstinline!year!} & qualitative ordinale & \passthrough{\lstinline!integer!} \\
\passthrough{\lstinline!acousticness!} & quantitative continue & \passthrough{\lstinline!numeric!} \\
\passthrough{\lstinline!duration!} & quantitative discrète & \passthrough{\lstinline!numeric!} \\
\passthrough{\lstinline!energy!} & quantitative continue & \passthrough{\lstinline!numeric!} \\
\passthrough{\lstinline!explicit!} & qualitative nominale & \passthrough{\lstinline!logical!} \\
\passthrough{\lstinline!key!} & qualitative nominale & \passthrough{\lstinline!factor!} \\
\passthrough{\lstinline!liveness!} & quantitative continue & \passthrough{\lstinline!numeric!} \\
\passthrough{\lstinline!loudness!} & quantitative continue & \passthrough{\lstinline!numeric!} \\
\passthrough{\lstinline!mode!} & qualitative nominale\footnote{On pose \passthrough{\lstinline!FALSE!} si la tonalité est mineure et \passthrough{\lstinline!TRUE!} si non.} & \passthrough{\lstinline!logical!} \\
\passthrough{\lstinline!tempo!} & quantitative continue & \passthrough{\lstinline!numeric!} \\
\passthrough{\lstinline!pop.class!} & qualitative ordinale & \passthrough{\lstinline!ordered!} \\
\bottomrule
\end{longtable}

\hypertarget{charger-les-jeux-de-donnuxe9es-dans-r}{%
\subsubsection{Charger les jeux de données dans R}\label{charger-les-jeux-de-donnuxe9es-dans-r}}

\begin{lstlisting}[language=R]
LoadDataset <- function(fname) {
    colclasses <- c(
        "integer", "numeric", "numeric",
        "numeric", "integer", "factor", "numeric",
        "numeric", "integer", "numeric", "factor"
    )
    dataframe <- read.csv(fname, colClasses = colclasses)
    dataframe$explicit <- as.logical(dataframe$explicit)
    dataframe$mode <- as.logical(dataframe$mode)
    dataframe$pop.class <- ordered(dataframe$pop.class)
    return(dataframe)
}
daf <- LoadDataset("dataset.csv")
str(daf)
\end{lstlisting}

\begin{lstlisting}
## 'data.frame':    10000 obs. of  11 variables:
##  $ year        : int  1998 1992 1973 1969 2008 2015 1935 1928 2013 1945 ...
##  $ acousticness: num  0.147 0.193 0.388 0.733 0.979 0.0742 0.99 0.995 0.000506 0.98 ...
##  $ duration    : num  148520 189800 289267 170267 438907 ...
##  $ energy      : num  0.74 0.389 0.856 0.454 0.494 0.766 0.42 0.211 0.53 0.106 ...
##  $ explicit    : logi  FALSE FALSE FALSE FALSE FALSE FALSE ...
##  $ key         : Factor w/ 12 levels "A","Ab","B","Bb",..: 7 5 1 10 11 11 2 6 12 12 ...
##  $ liveness    : num  0.0452 0.154 0.139 0.0889 0.123 0.0827 0.13 0.106 0.0477 0.237 ...
##  $ loudness    : num  -8.16 -11.64 -8.4 -8.12 -10.65 ...
##  $ mode        : logi  FALSE FALSE FALSE FALSE FALSE FALSE ...
##  $ tempo       : num  157.1 85.3 101.3 82.4 156.3 ...
##  $ pop.class   : Ord.factor w/ 4 levels "A"<"B"<"C"<"D": 3 3 3 3 3 1 4 4 2 4 ...
\end{lstlisting}

\hypertarget{analyses-unidimensionnelles}{%
\subsection{Analyses unidimensionnelles}\label{analyses-unidimensionnelles}}

\hypertarget{une-variable-qualitative---pop.class}{%
\subsubsection{\texorpdfstring{Une variable qualitative - \texttt{pop.class}}{Une variable qualitative - pop.class}}\label{une-variable-qualitative---pop.class}}

\begin{lstlisting}[language=R]
summary(daf$pop.class)
\end{lstlisting}

\begin{lstlisting}
##    A    B    C    D 
##  940 2874 3038 3148
\end{lstlisting}

Il existe 4 niveaux de popularité (modalités). Commencer par \passthrough{\lstinline!A!} est le plus populaire et décroissant avec \passthrough{\lstinline!B!}, \passthrough{\lstinline!C!}, \passthrough{\lstinline!D!}.

\begin{lstlisting}[language=R]
pop_class_table <- table(daf$pop.class)
print(label_percent()(c(pop_class_table) / sum(pop_class_table)), quote = F)
\end{lstlisting}

\begin{lstlisting}
##     A     B     C     D 
##  9.4% 28.7% 30.4% 31.5%
\end{lstlisting}

\begin{figure}

{\centering \includegraphics{report_files/figure-latex/pop-class-table-1} 

}

\caption{Diagramme en barre de popularité}\label{fig:pop-class-table}
\end{figure}

On peut noter que dans cet ensemble de données, la plupart des chansons ne sont pas populaires (\(31,5%
\)).
Plus le niveau de popularité est élevé, moins les chansons peuvent atteindre ce niveau.

\hypertarget{une-variable-quantitative---acousticness}{%
\subsubsection{\texorpdfstring{Une variable quantitative - \texttt{acousticness}}{Une variable quantitative - acousticness}}\label{une-variable-quantitative---acousticness}}

\hypertarget{ruxe9sumuxe9}{%
\paragraph{Résumé}\label{ruxe9sumuxe9}}

\begin{lstlisting}[language=R]
summary(daf$acousticness)
\end{lstlisting}

\begin{lstlisting}
##    Min. 1st Qu.  Median    Mean 3rd Qu.    Max. 
##  0.0000  0.0961  0.5085  0.4990  0.8930  0.9960
\end{lstlisting}

D'après le résultat ci-dessus, on a:

\begin{itemize}
\tightlist
\item
  Le premier quartile \(q_{0.25}\) est \(0.0961\)
\item
  Le deuxième quartile \(q_{0.5}\) est \(0.5085\)
\item
  Le troisième quartile \(q_{0.75}\) est \(0.8930\)
\end{itemize}

\hypertarget{distribution}{%
\paragraph{Distribution}\label{distribution}}

\begin{lstlisting}[language=R]
skewness(daf$acousticness)
\end{lstlisting}

\begin{lstlisting}
## [1] -0.01816556
\end{lstlisting}

Étant donné que son skewness est approximativement 0, nous pouvons conclure que
l'ensemble de données est centré autour de sa médiane.

\begin{lstlisting}[language=R]
kurtosis(daf$acousticness)
\end{lstlisting}

\begin{lstlisting}
## [1] -1.613205
\end{lstlisting}

Du fait que son kurtosis est inférieur à \(-1,2\) (le kurtosis de la distribution uniforme\footnote{\url{https://en.wikipedia.org/wiki/Kurtosis\#Other_well-known_distributions}}),
sa distribution aura la forme d'une vallée (car la distribution uniforme est déjà une ligne).

\begin{figure}

{\centering \includegraphics{report_files/figure-latex/daf-acousticness-hist-1} 

}

\caption{Histogramme d'acoustique des chansons}\label{fig:daf-acousticness-hist}
\end{figure}

Vous pouvez voir toutes les caractéristiques mentionnées ci-dessus dans la figure \ref{fig:daf-acousticness-hist}.

\hypertarget{analyses-bidimensionnelles}{%
\subsection{Analyses bidimensionnelles}\label{analyses-bidimensionnelles}}

\hypertarget{entre-une-variable-quantitative-et-une-qualitative}{%
\subsubsection{Entre une variable quantitative et une qualitative}\label{entre-une-variable-quantitative-et-une-qualitative}}

Dans cette partie, nous réutiliserons et analyserons les 2 variables précédentes (\passthrough{\lstinline!pop.class!} et \passthrough{\lstinline!acousticness!}).

\hypertarget{repruxe9sentation-graphique}{%
\paragraph{Représentation graphique}\label{repruxe9sentation-graphique}}

\begin{figure}

{\centering \includegraphics{report_files/figure-latex/daf-acousticness-popclass-boxplot-1} 

}

\caption{Boxplot parallèles de la relation entre acoustique et popularité}\label{fig:daf-acousticness-popclass-boxplot}
\end{figure}

Notez à partir de notre graphique, les cases varient d'un facteur à l'autre,
nous concluons que l'acoustique et la popularité sont liées l'une à l'autre.

De plus, \(75\%\) des chansons populaires ont une acoustique inférieure à \(0,5\) tandis que celle de presque \(100\%\) des chansons les moins populaires est supérieure à \(0,5\).
De l'autre côté, selon la partie précédente, la distribution des chansons avec l'acoustique est symétrique autour de sa médiane (ce qui indique qu'il y a presque le même nombre de chansons de 2 types).
Il est démontrable que les gens aiment les chansons électroniques.

\hypertarget{indice-de-liaison}{%
\paragraph{Indice de liaison}\label{indice-de-liaison}}

\begin{lstlisting}[language=R]
eta2(daf$acousticness, daf$pop.class)
\end{lstlisting}

\begin{lstlisting}
## [1] 0.3673706
\end{lstlisting}

Avec \(c_{y|x} \approx 0,4\), il existe une légère relation entre deux variables.

\hypertarget{volume-energy}{%
\subsubsection{Entre deux variables quantitatives}\label{volume-energy}}

Dans cette partie, nous étudions la relation entre le volume et l'énergie.

\hypertarget{repruxe9sentation-graphique-1}{%
\paragraph{Représentation graphique}\label{repruxe9sentation-graphique-1}}

\begin{figure}

{\centering \includegraphics{report_files/figure-latex/daf-point-energy-loudness-1} 

}

\caption{Nuage de points et droite de régression entre l'énergie et le volume}\label{fig:daf-point-energy-loudness}
\end{figure}

\hypertarget{indices-de-liaison}{%
\paragraph{Indices de liaison}\label{indices-de-liaison}}

\begin{lstlisting}[language=R]
cor(daf$energy, daf$loudness)
\end{lstlisting}

\begin{lstlisting}
## [1] 0.7744876
\end{lstlisting}

Avec cette valeur de corrélation et à partir de la figure \ref{fig:daf-point-energy-loudness},
nous en déduisons qu'il existe un lien fort entre le volume d'une chanson et son énergie:
plus le volume d'une chanson est forte, plus elle a d'énergie.

\hypertarget{analyse-en-composantes-principales-acp}{%
\section{Analyse en composantes principales (ACP)}\label{analyse-en-composantes-principales-acp}}

Tout d'abord, nous supprimerons toutes les colonnes qualitatives de
l'ensemble de données (qui sont \passthrough{\lstinline!year!}, \passthrough{\lstinline!explicit!}, \passthrough{\lstinline!key!}, \passthrough{\lstinline!mode!} et \passthrough{\lstinline!pop.class!})

\begin{lstlisting}[language=R]
dafacp <- select_if(daf, is.numeric)
dafacp <- dafacp[,-1] # remove the first column of `dafacp` (which is `year`)
\end{lstlisting}

\hypertarget{choix-du-type-dacp-ruxe9alisuxe9}{%
\subsection{Choix du type d'ACP réalisé}\label{choix-du-type-dacp-ruxe9alisuxe9}}

\begin{figure}

{\centering \subfloat[Des données brutes\label{fig:boxplot-comparison-1}]{\includegraphics[width=0.5\linewidth]{report_files/figure-latex/boxplot-comparison-1} }\subfloat[Des données centrées réduites\label{fig:boxplot-comparison-2}]{\includegraphics[width=0.5\linewidth]{report_files/figure-latex/boxplot-comparison-2} }

}

\caption{Boxplots des données quantitatives}\label{fig:boxplot-comparison}
\end{figure}

Ici, le choix de faire une \textbf{ACP centrée réduite} s'impose pour deux raisons :

\begin{itemize}
\tightlist
\item
  Les données des différentes variables ne sont pas du tout à la même échelle comme on peut voir dans le figure \ref{fig:boxplot-comparison}.
\item
  Elles ont des unités différentes.
\end{itemize}

\hypertarget{factominer}{%
\subsection{FactoMineR}\label{factominer}}

Nous allons, à partir d'ici, utiliser la librairie \passthrough{\lstinline!FactoMineR!} pour effectuer l'ACP en incluant directement l'étape de centrer réduire faite plus haut.
L'idée est de se dire que parmi nos 6 variables, nous avons de l'information redondante et donc de passer en dimension plus faible
(2 ou 3 pour faciliter la représentation) grâce à des méta-variables. Seulement, nous ne pouvons choisir ces méta-variables au hasard,
elles doivent correspondre aux directions selon lesquelles on a le plus de variabilité. La variabilité totale, qui est l'inertie,
est répartie entre les 6 dimensions.

\begin{lstlisting}[language=R]
result_acp <- PCA(daf, scale.unit = TRUE, ncp = 6, quali.sup = c(1, 5, 6, 9, 11), graph = FALSE)
\end{lstlisting}

Avec FactoMineR, nous effectuons ici une ACP à 6 dimensions (\passthrough{\lstinline!ncp = 6!} pour nos 6 variables quantitatives) et
nous rajoutons les autres 5 variables comme variables qualitatives supplémentaires
(elles nous serviront lors de l'interprétation).

\begin{lstlisting}[language=R]
result_acp$eig[,"eigenvalue"]
\end{lstlisting}

\begin{lstlisting}
##    comp 1    comp 2    comp 3    comp 4    comp 5    comp 6 
## 2.5169626 1.0407603 0.9768164 0.8844294 0.4183029 0.1627285
\end{lstlisting}

\begin{figure}

{\centering \includegraphics{report_files/figure-latex/bar-per-1} 

}

\caption{Pourcentages cumulés d'inertie portés par chaque axe}\label{fig:bar-per}
\end{figure}

Regardons plus particulièrement les valeurs propres et les pourcentages d'inertie associés à chaque dimension.
Comme on s'y attend, l'ACP classe les dimensions de la plus influente à la moins influente.
Ici, nous choisissons de garder \(75\%\) de variabilité (ce qui est déja un très bon seuil).
Ce dernier est atteint par la dimension 3 comme l'indique la droite bleu sur le graphe de pourcentages cumulés d'inertie.
Nous prenons donc les 3 premières dimensions.

\hypertarget{etude-des-individus}{%
\subsection{Etude des individus}\label{etude-des-individus}}

Pour étudier les individus, on se rappelle qu'une ligne du tableau de départ correspond à un individu qu'on veut représenter par un point sur un graphique.
Ainsi, à l'issue des 10000 lignes nous auront notre nuage de 10000 individus points représentant les individus. Cela aurait été simple avec deux varibles x et y, le point serait représenter dans un plan à 2 dimensions (de même avec 3 variables en 3 dimensions).
Mais nous avons 6 variables qui définissent chaque individu ici, nous sommes donc passées par l'ACP qui a réduit cela à 3 dimensions. Nous
allons regarder nos individus dans ces trois dimensions.

\begin{figure}

{\centering \includegraphics[width=2\linewidth]{report_files/figure-latex/projection-pop-12-1} 

}

\caption{Projection des chansons selon leur popularité dans le plan des axes principaux 1 et 2}\label{fig:projection-pop-12}
\end{figure}
\begin{figure}

{\centering \includegraphics{report_files/figure-latex/projection-pop-13-1} 

}

\caption{Projection des chansons selon leur popularité dans le plan des axes principaux 1 et 3}\label{fig:projection-pop-13}
\end{figure}

On a plusieurs nuages de points et on voit une tendance qui ressortent par lecture graphique. A partir des figures
\ref{fig:projection-pop-12} et \ref{fig:projection-pop-13}, on observe que la dimension 1 semble être assez représentative de la popularité d'une chanson:
bleu quand les coordonnées sont diminués sur la dimension 1 et autre couleurs quand elles augmentent.

\hypertarget{etude-des-variables}{%
\subsection{Etude des variables}\label{etude-des-variables}}

On représente le graphe des corrélations des variables:

\begin{figure}

{\centering \subfloat[les dimensions 1 et 2\label{fig:corvar-acp-12-1}]{\includegraphics[width=0.5\linewidth]{report_files/figure-latex/corvar-acp-12-1} }\subfloat[les dimensions 1 et 3\label{fig:corvar-acp-12-2}]{\includegraphics[width=0.5\linewidth]{report_files/figure-latex/corvar-acp-12-2} }

}

\caption{Graphe de corrélation entre les variables}\label{fig:corvar-acp-12}
\end{figure}

D'après la lecture de la figure 9, nous avons les relations suivantes:

\begin{itemize}
\tightlist
\item
  En suivant la direction positive de l'axe de dim 1, nous avons une augmentation de l'énergie et du volume de la chanson (ce qui est également logique car les 2 variables ont une relation forte selon \protect\hyperlink{volume-energy}{analyses bidimensionnelles entre le volume et l'énergie})
\item
  Dans le sens opposé, l'acoustique de la chanson monte. Il est également cohérent avec le fait que l'acoustique d'une chanson est généralement inversement proportionnelle à son énergie et à son volume.
\item
  La durée de la chanson pourrait être représentée à la fois par la dimension 2 et la dimension 3.
\item
  La même chose est vraie pour la vivacité (\passthrough{\lstinline!liveness!}).
\end{itemize}

On peut retrouver ces liaisons en observant la figure \ref{fig:corr-var-dim}.

\begin{figure}

{\centering \includegraphics{report_files/figure-latex/corr-var-dim-1} 

}

\caption{Le graphe de corrélation entre variables et dimensions}\label{fig:corr-var-dim}
\end{figure}

\end{document}
