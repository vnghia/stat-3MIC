% Options for packages loaded elsewhere
\PassOptionsToPackage{unicode}{hyperref}
\PassOptionsToPackage{hyphens}{url}
%
\documentclass[
  12pt,
  xcolor = usenames,dvipsnames]{article}
\usepackage{amsmath,amssymb}
\usepackage{lmodern}
\usepackage{iftex}
\ifPDFTeX
  \usepackage[T1]{fontenc}
  \usepackage[utf8]{inputenc}
  \usepackage{textcomp} % provide euro and other symbols
\else % if luatex or xetex
  \usepackage{unicode-math}
  \defaultfontfeatures{Scale=MatchLowercase}
  \defaultfontfeatures[\rmfamily]{Ligatures=TeX,Scale=1}
\fi
% Use upquote if available, for straight quotes in verbatim environments
\IfFileExists{upquote.sty}{\usepackage{upquote}}{}
\IfFileExists{microtype.sty}{% use microtype if available
  \usepackage[]{microtype}
  \UseMicrotypeSet[protrusion]{basicmath} % disable protrusion for tt fonts
}{}
\makeatletter
\@ifundefined{KOMAClassName}{% if non-KOMA class
  \IfFileExists{parskip.sty}{%
    \usepackage{parskip}
  }{% else
    \setlength{\parindent}{0pt}
    \setlength{\parskip}{6pt plus 2pt minus 1pt}}
}{% if KOMA class
  \KOMAoptions{parskip=half}}
\makeatother
\usepackage{fancyvrb}
\usepackage{xcolor}
\IfFileExists{xurl.sty}{\usepackage{xurl}}{} % add URL line breaks if available
\IfFileExists{bookmark.sty}{\usepackage{bookmark}}{\usepackage{hyperref}}
\hypersetup{
  hidelinks,
  pdfcreator={LaTeX via pandoc}}
\urlstyle{same} % disable monospaced font for URLs
\VerbatimFootnotes % allow verbatim text in footnotes
\usepackage[margin=1in]{geometry}
\usepackage{listings}
\newcommand{\passthrough}[1]{#1}
\lstset{defaultdialect=[5.3]Lua}
\lstset{defaultdialect=[x86masm]Assembler}
\usepackage{longtable,booktabs,array}
\usepackage{calc} % for calculating minipage widths
% Correct order of tables after \paragraph or \subparagraph
\usepackage{etoolbox}
\makeatletter
\patchcmd\longtable{\par}{\if@noskipsec\mbox{}\fi\par}{}{}
\makeatother
% Allow footnotes in longtable head/foot
\IfFileExists{footnotehyper.sty}{\usepackage{footnotehyper}}{\usepackage{footnote}}
\makesavenoteenv{longtable}
\usepackage{graphicx}
\makeatletter
\def\maxwidth{\ifdim\Gin@nat@width>\linewidth\linewidth\else\Gin@nat@width\fi}
\def\maxheight{\ifdim\Gin@nat@height>\textheight\textheight\else\Gin@nat@height\fi}
\makeatother
% Scale images if necessary, so that they will not overflow the page
% margins by default, and it is still possible to overwrite the defaults
% using explicit options in \includegraphics[width, height, ...]{}
\setkeys{Gin}{width=\maxwidth,height=\maxheight,keepaspectratio}
% Set default figure placement to htbp
\makeatletter
\def\fps@figure{htbp}
\makeatother
\setlength{\emergencystretch}{3em} % prevent overfull lines
\providecommand{\tightlist}{%
  \setlength{\itemsep}{0pt}\setlength{\parskip}{0pt}}
\setcounter{secnumdepth}{5}
\usepackage{setspace}
\lstset{ 
  language=R,                     % the language of the code
  basicstyle=\small\ttfamily, % the size of the fonts that are used for the code
  stepnumber=1,                   % the step between two line-numbers. If it is 1, each line
                                  % will be numbered
  numbersep=5pt,                  % how far the line-numbers are from the code
  backgroundcolor=\color{cyan!5},  % choose the background color. You must add \usepackage{color}
  showspaces=false,               % show spaces adding particular underscores
  showstringspaces=false,         % underline spaces within strings
  showtabs=false,                 % show tabs within strings adding particular underscores
  frame=single,                   % adds a frame around the code
  rulecolor=\color{black},        % if not set, the frame-color may be changed on line-breaks within not-black text (e.g. commens (green here))
  tabsize=2,                      % sets default tabsize to 2 spaces
  captionpos=b,                   % sets the caption-position to bottom
  breaklines=true,                % sets automatic line breaking
  breakatwhitespace=false,        % sets if automatic breaks should only happen at whitespace
  keywordstyle=\color{RoyalBlue},      % keyword style
  commentstyle=\color{Green},   % comment style
  stringstyle=\color{Orange},      % string literal style
}
\makeatletter
\renewcommand\paragraph{\@startsection{paragraph}{4}{\z@}%
        {-2.5ex\@plus -1ex \@minus -.25ex}%
        {1.25ex \@plus .25ex}%
        {\normalfont\normalsize\bfseries}}
\makeatother
\setcounter{secnumdepth}{4}
\ifLuaTeX
  \usepackage{selnolig}  % disable illegal ligatures
\fi

\author{}
\date{\vspace{-2.5em}}

\begin{document}

\onehalfspacing

\pagenumbering{gobble}

%\begin{titlepage}
\vspace*{\fill}
\begin{center}
\LARGE{\textbf{Analyse descriptive du jeu de données Spotify}}\\
\Large{\textbf{Projet en Statistique descriptive}}\\
\vspace*{1\baselineskip}
\Large{\textbf{Membres}}\\
LOULIDI Younes\\
PHAM Tuan Kiet\\
VO Van Nghia\\
\vfill % equivalent to \vspace{\fill}
\vspace*{\fill}
\Large{\textbf{Date}}\\
17 Mars, 2021
\end{center}
% \end{titlepage}

\newpage
\doublespacing

\hypersetup{linkcolor = black}
\newpage
\renewcommand{\contentsname}{Table des matières}
\pagenumbering{roman}
\tableofcontents
\addcontentsline{toc}{section}{\contentsname}

\newpage
\pagenumbering{arabic}


\hypertarget{statistiques-descriptives-unidimensionnelle-et-bidimensionnelle}{%
\section{Statistiques descriptives unidimensionnelle et
bidimensionnelle}\label{statistiques-descriptives-unidimensionnelle-et-bidimensionnelle}}

\hypertarget{la-nature-des-jeux-de-donnuxe9es}{%
\subsection{La nature des jeux de
données}\label{la-nature-des-jeux-de-donnuxe9es}}

\hypertarget{des-jeux-de-donnuxe9es}{%
\subsubsection{Des jeux de données}\label{des-jeux-de-donnuxe9es}}

Ces jeux de données se composent de 10000 chansons extraites de la base
de données Spotify. Chaque ligne contient 11 variables statistiques
comme suit:

\begin{itemize}
\tightlist
\item
  \passthrough{\lstinline!year!}: année de sortie du morceau,
\item
  \passthrough{\lstinline!acousticness!}: métrique relative interne de
  l'acoustique morceau,
\item
  \passthrough{\lstinline!duration!}: durée du morceau en millisecondes
  (ms),
\item
  \passthrough{\lstinline!energy!}: métrique relative interne de
  l'intensité, des rythmes du morceau,
\item
  \passthrough{\lstinline!explicit!}: vaut 1 si le morceau contient des
  vulgarités, et 0 sinon,
\item
  \passthrough{\lstinline!key!}: tonalité en début de morceau,
\item
  \passthrough{\lstinline!liveness!}: proportion du morceau où l'on
  entend un public,
\item
  \passthrough{\lstinline!loudness!}: mesure relative du volume du
  morceau (en décibels, dB)
\item
  \passthrough{\lstinline!mode!}: mode du morceau (0 si la tonalité est
  mineure, et 1 si la tonalité est majeure),
\item
  \passthrough{\lstinline!tempo!}: le tempo du morceau, en battement par
  minute (bpm),
\item
  \passthrough{\lstinline!pop.class!}: la popularité du morceau.
\end{itemize}

\hypertarget{des-variables-statistiques}{%
\subsubsection{Des variables
statistiques}\label{des-variables-statistiques}}

Ici, nous précisons la nature de chaque variable et son format dans R.

\begin{longtable}[]{@{}llr@{}}
\toprule
Nom de variable statistique & Type de variable & Format dans R \\
\midrule
\endhead
\passthrough{\lstinline!year!} & qualitative ordinale &
\passthrough{\lstinline!integer!} \\
\passthrough{\lstinline!acousticness!} & quantitative continue &
\passthrough{\lstinline!numeric!} \\
\passthrough{\lstinline!duration!} & quantitative continue\footnote{On
  choisi son nature est de quantitative continue parce que.} &
\passthrough{\lstinline!numeric!} \\
\passthrough{\lstinline!energy!} & quantitative continue &
\passthrough{\lstinline!numeric!} \\
\passthrough{\lstinline!explicit!} & qualitative nominale &
\passthrough{\lstinline!logical!} \\
\passthrough{\lstinline!key!} & qualitative nominale &
\passthrough{\lstinline!factor!} \\
\passthrough{\lstinline!liveness!} & quantitative continue &
\passthrough{\lstinline!numeric!} \\
\passthrough{\lstinline!loudness!} & quantitative continue &
\passthrough{\lstinline!numeric!} \\
\passthrough{\lstinline!mode!} & qualitative nominale\footnote{On pose
  \passthrough{\lstinline!FALSE!} si la tonalité est mineure et
  \passthrough{\lstinline!TRUE!} si non.} &
\passthrough{\lstinline!logical!} \\
\passthrough{\lstinline!tempo!} & quantitative continue &
\passthrough{\lstinline!numeric!} \\
\passthrough{\lstinline!pop.class!} & qualitative ordinale &
\passthrough{\lstinline!factor!} \\
\bottomrule
\end{longtable}

\hypertarget{charger-les-jeux-de-donnuxe9es-dans-r}{%
\subsubsection{Charger les jeux de données dans
R}\label{charger-les-jeux-de-donnuxe9es-dans-r}}

\begin{lstlisting}[language=R]
LoadDataset <- function(fname) {
    colclasses <- c(
        "integer", "numeric", "numeric",
        "numeric", "integer", "factor", "numeric",
        "numeric", "integer", "numeric", "factor"
    )
    dataframe <- read.csv(fname, colClasses = colclasses)
    dataframe$explicit <- as.logical(dataframe$explicit)
    dataframe$mode <- as.logical(dataframe$mode)
    return(dataframe)
}
daf <- LoadDataset("dataset.csv")
str(daf)
\end{lstlisting}

\begin{lstlisting}
## 'data.frame':    10000 obs. of  11 variables:
##  $ year        : int  1998 1992 1973 1969 2008 2015 1935 1928 2013 1945 ...
##  $ acousticness: num  0.147 0.193 0.388 0.733 0.979 0.0742 0.99 0.995 0.000506 0.98 ...
##  $ duration    : num  148520 189800 289267 170267 438907 ...
##  $ energy      : num  0.74 0.389 0.856 0.454 0.494 0.766 0.42 0.211 0.53 0.106 ...
##  $ explicit    : logi  FALSE FALSE FALSE FALSE FALSE FALSE ...
##  $ key         : Factor w/ 12 levels "A","Ab","B","Bb",..: 7 5 1 10 11 11 2 6 12 12 ...
##  $ liveness    : num  0.0452 0.154 0.139 0.0889 0.123 0.0827 0.13 0.106 0.0477 0.237 ...
##  $ loudness    : num  -8.16 -11.64 -8.4 -8.12 -10.65 ...
##  $ mode        : logi  FALSE FALSE FALSE FALSE FALSE FALSE ...
##  $ tempo       : num  157.1 85.3 101.3 82.4 156.3 ...
##  $ pop.class   : Factor w/ 4 levels "A","B","C","D": 3 3 3 3 3 1 4 4 2 4 ...
\end{lstlisting}

\hypertarget{analyses-unidimensionnelles}{%
\subsection{Analyses
unidimensionnelles}\label{analyses-unidimensionnelles}}

\hypertarget{une-variable-qualitative---pop.class}{%
\subsubsection{\texorpdfstring{Une variable qualitative -
\texttt{pop.class}}{Une variable qualitative - pop.class}}\label{une-variable-qualitative---pop.class}}

\begin{lstlisting}[language=R]
summary(daf$pop.class)
\end{lstlisting}

\begin{lstlisting}
##    A    B    C    D 
##  940 2874 3038 3148
\end{lstlisting}

Il existe 4 niveaux de popularité (modalités). Commencer par
\passthrough{\lstinline!A!} est le plus populaire et décroissant avec
\passthrough{\lstinline!B!}, \passthrough{\lstinline!C!},
\passthrough{\lstinline!D!}.

\begin{lstlisting}[language=R]
pop_class_table <- table(daf$pop.class)
print(label_percent()(c(pop_class_table) / sum(pop_class_table)), quote = F)
\end{lstlisting}

\begin{lstlisting}
##     A     B     C     D 
##  9.4% 28.7% 30.4% 31.5%
\end{lstlisting}

\begin{figure}
\centering
\includegraphics{report_files/figure-latex/pop_class_table-1.pdf}
\caption{Diagramme en barre de popularité}
\end{figure}

On peut noter que dans cet ensemble de données, la plupart des chansons
ne sont pas populaires (\(31,5%
\)). Plus le niveau de popularité est élevé, moins les chansons peuvent
atteindre ce niveau.

\hypertarget{une-variable-quantitative---acousticness}{%
\subsubsection{\texorpdfstring{Une variable quantitative -
\texttt{acousticness}}{Une variable quantitative - acousticness}}\label{une-variable-quantitative---acousticness}}

\hypertarget{ruxe9sumuxe9}{%
\paragraph{Résumé}\label{ruxe9sumuxe9}}

\begin{lstlisting}[language=R]
summary(daf$acousticness)
\end{lstlisting}

\begin{lstlisting}
##    Min. 1st Qu.  Median    Mean 3rd Qu.    Max. 
##  0.0000  0.0961  0.5085  0.4990  0.8930  0.9960
\end{lstlisting}

D'après le résultat ci-dessus, on a:

\begin{itemize}
\tightlist
\item
  Le premier quartile \(q_{0.25}\) est \(0.0961\)
\item
  Le deuxième quartile \(q_{0.5}\) est \(0.5085\)
\item
  Le troisième quartile \(q_{0.75}\) est \(0.8930\)
\end{itemize}

\hypertarget{distribution}{%
\paragraph{Distribution}\label{distribution}}

\begin{lstlisting}[language=R]
skewness(daf$acousticness)
\end{lstlisting}

\begin{lstlisting}
## [1] -0.01816556
\end{lstlisting}

Étant donné que son skewness est approximativement 0, nous pouvons
conclure que l'ensemble de données est centré autour de sa médiane.

\begin{lstlisting}[language=R]
kurtosis(daf$acousticness)
\end{lstlisting}

\begin{lstlisting}
## [1] -1.613205
\end{lstlisting}

Du fait que son kurtosis est inférieur à \(-1,2\) (le kurtosis de la
distribution uniforme\footnote{\url{https://en.wikipedia.org/wiki/Kurtosis\#Other_well-known_distributions}}),
sa distribution aura la forme d'une vallée (car la distribution uniforme
est déjà une ligne).

\begin{figure}
\centering
\includegraphics{report_files/figure-latex/daf_acousticness_boxplot-1.pdf}
\caption{Histogramme d'acoustique des chansons}
\end{figure}

\hypertarget{analyses-bidimensionnelles}{%
\subsection{Analyses
bidimensionnelles}\label{analyses-bidimensionnelles}}

\hypertarget{entre-une-variable-quantitative-et-une-qualitative}{%
\subsubsection{Entre une variable quantitative et une
qualitative}\label{entre-une-variable-quantitative-et-une-qualitative}}

Dans cette partie, nous réutiliserons et analyserons les 2 variables
précédentes (\passthrough{\lstinline!pop.class!} et
\passthrough{\lstinline!acousticness!}).

\hypertarget{repruxe9sentation-graphique}{%
\paragraph{Représentation graphique}\label{repruxe9sentation-graphique}}

\begin{figure}
\centering
\includegraphics{report_files/figure-latex/daf_acousticness_popclass_boxplot-1.pdf}
\caption{Boxplot parallèles de la relation entre acoustique et
popularité}
\end{figure}

Notez à partir de notre graphique, les cases varient d'un facteur à
l'autre, nous concluons que l'acoustique et la popularité sont liées
l'une à l'autre.

De plus, \(75\%\) des chansons populaires ont une acoustique inférieure
à \(0,5\) tandis que celle de presque \(100\%\) des chansons les moins
populaires est supérieure à \(0,5\). De l'autre côté, selon la partie
précédente, la distribution des chansons avec l'acoustique est
symétrique autour de sa médiane (ce qui indique qu'il y a presque le
même nombre de chansons de 2 types). Il est démontrable que les gens
aiment les chansons électroniques.

\hypertarget{indice-de-liaison}{%
\paragraph{Indice de liaison}\label{indice-de-liaison}}

\begin{lstlisting}[language=R]
eta2(daf$acousticness, daf$pop.class)
\end{lstlisting}

\begin{lstlisting}
## [1] 0.3673706
\end{lstlisting}

Avec \(c_{y|x} \approx 0,4\), il existe une légère relation entre deux
variables.

\end{document}
